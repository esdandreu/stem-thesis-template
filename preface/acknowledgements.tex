% Special indentation for acknowledgments.
\setlength{\parskip}{1em}
\setlength{\parindent}{0em}

\noindent
The branch of technology that deals with the design, construction, operation,
and application of robots, as well as computer systems for their control,
sensory feedback, and information processing is robotics. These technologies
deal with automated machines that can take the place of humans in dangerous
environments or manufacturing processes, or resemble humans in appearance,
behaviour, or cognition. Many of today's robots are inspired by nature
contributing to the field of bio-inspired robotics. These robots have also
created a newer branch of robotics: soft robotics.

From the time of ancient civilization there have been many accounts of
user-configurable automated devices and even automata resembling animals and
humans, designed primarily as entertainment. As mechanical techniques developed
through the Industrial age, there appeared more practical applications such as
automated machines, remote-control and wireless remote-control.

The term comes from a Slavic root, robot-, with meanings associated with
labour. The word 'robot' was first used to denote a fictional humanoid in a
1920 Czech-language play R.U.R. (Rossumovi Univerzální Roboti - Rossum's
Universal Robots) by Karel Čapek, though it was Karel's brother Josef Čapek who
was the word's true inventor. Electronics evolved into the driving force of
development with the advent of the first electronic autonomous robots created
by William Grey Walter in Bristol, England in 1948, as well as Computer
Numerical Control (CNC) machine tools in the late 1940s by John T. Parsons and
Frank L. Stulen. The first commercial, digital and programmable robot was built
by George Devol in 1954 and was named the Unimate. It was sold to General
Motors in 1961 where it was used to lift pieces of hot metal from die casting
machines at the Inland Fisher Guide Plant in the West Trenton section of Ewing
Township, New Jersey.
